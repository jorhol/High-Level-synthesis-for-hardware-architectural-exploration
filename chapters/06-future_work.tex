\chapter{Future work}
\label{chp:futurework} 
The primary objectives of this project, namely to create an initial framework for hardware architectural exploration using \gls{hls}, were not met. The reasons for this were multiple unexpected, time-consuming problems along the way, described in detail in section \ref{sec:encprob}. Also, the goals of the project, set in the beginning of the working period, might have been slightly exaggerated.
If this project should be extended into a master thesis or a project at a later stage, the following issues needs to be handled:
\begin{itemize}
\item The C compiler needs to be altered, so that the list of input variables and return values in the \textit{main}-function of the C program can be of other formats than the standard \textit{int main(int argc, char *argv[])}.
\item Pointers in the C program needs to be transformed into signals instead of a memory location inside the memory controller. As there are only one module in the output Verilog, there are no point in wondering in which module the pointed variable will be used. 
\item The memory controller module, as well as all other unused modules, need to be removed from the Verilog generation. This module is not needed when there is no hybrid-flow and pointers no longer is stored in global memory.
\end{itemize}

There are two possible alternatives to fix these problems:
\begin{enumerate}
\item Post-processing of the Verilog generated by LegUp in order to remove unwanted code and adapt the code into a format suitable for synthesis towards an \gls{asic} implementation.
\item Pre-processing, i.e. altering the LegUp and \gls{llvm} libraries that are used for compilation of the source code and Verilog generation so that the output Verilog are synthesizable to \gls{asic} applications.
\end{enumerate}

The post-processing solution is clearly the simplest alternative, as it only requires a parser that can remove or reformat code output. This alternative is however very uncertain with respect to the quality of results. The question is whether the output Verilog contains enough information to transform it into a usable Verilog code.
The alternative of altering the LegUp and \gls{llvm} libraries, offers much greater opportunities for you to get a functional and correct Verilog output. The downside of this alternative is that the LegUp libraries are quite large, and searching, understanding and altering large parts of the code can be time-consuming. 

Adding the flag \textit{-ffreestanding} to the variable CLANG\_FLAG in the file Makefile.config fixes this problem by not issuing the error. Return values still need to be single variable, pointer or struct (struct not supported without altsyncram)

Extending the Verilog generating part of LegUp into outputting Verilog compliant to Nordic's \gls{ddvc}.